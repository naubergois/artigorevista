%% This is file `elsarticle-template-1-num.tex',
%%
%% Copyright 2009 Elsevier Ltd
%%
%% This file is part of the 'Elsarticle Bundle'.
%% ---------------------------------------------
%%
%% It may be distributed under the conditions of the LaTeX Project Public
%% License, either version 1.2 of this license or (at your option) any
%% later version.  The latest version of this license is in
%%    http://www.latex-project.org/lppl.txt
%% and version 1.2 or later is part of all distributions of LaTeX
%% version 1999/12/01 or later.
%%
%% Template article for Elsevier's document class `elsarticle'
%% with numbered style bibliographic references
%%
%% $Id: elsarticle-template-1-num.tex 149 2009-10-08 05:01:15Z rishi $
%% $URL: http://lenova.river-valley.com/svn/elsbst/trunk/elsarticle-template-1-num.tex $
%%
%\documentclass[preprint,12pt]{elsarticle}

%% Use the option review to obtain double line spacing
%% \documentclass[preprint,review,12pt]{elsarticle}

%% Use the options 1p,twocolumn; 3p; 3p,twocolumn; 5p; or 5p,twocolumn
%% for a journal layout:
%% \documentclass[final,1p,times]{elsarticle}
%\documentclass[final,1p,times,twocolumn]{elsarticle}
%\documentclass[final,3p,times]{elsarticle}
\documentclass[final,3p,times,twocolumn]{elsarticle}
%% \documentclass[final,5p,times]{elsarticle}
%% \documentclass[final,5p,times,twocolumn]{elsarticle}

%% The graphicx package provides the includegraphics command.
\usepackage{graphicx}
%% The amssymb package provides various useful mathematical symbols
\usepackage{amssymb}
\usepackage{rotating}


%% The amsthm package provides extended theorem environments
%% \usepackage{amsthm}

%% The lineno packages adds line numbers. Start line numbering with
%% \begin{linenumbers}, end it with \end{linenumbers}. Or switch it on
%% for the whole article with \linenumbers after \end{frontmatter}.
\usepackage{lineno}

%% natbib.sty is loaded by default. However, natbib options can be
%% provided with \biboptions{...} command. Following options are
%% valid:

%%   round  -  round parentheses are used (default)
%%   square -  square brackets are used   [option]
%%   curly  -  curly braces are used      {option}
%%   angle  -  angle brackets are used    <option>
%%   semicolon  -  multiple citations separated by semi-colon
%%   colon  - same as semicolon, an earlier confusion
%%   comma  -  separated by comma
%%   numbers-  selects numerical citations
%%   super  -  numerical citations as superscripts
%%   sort   -  sorts multiple citations according to order in ref. list
%%   sort&compress   -  like sort, but also compresses numerical citations
%%   compress - compresses without sorting
%%
%% \biboptions{comma,round}

% \biboptions{}

\usepackage{import}
\usepackage{amsmath}
\usepackage{multirow}
\usepackage{graphicx,url}
\usepackage{placeins}
\usepackage{adjustbox}
\usepackage[english]{babel}
\usepackage{lipsum}
\usepackage{multicol}
\usepackage{listings}
\usepackage[svgnames]{xcolor} 
\usepackage{caption}
\usepackage{amsmath}
\usepackage{calc} 
\usepackage{array,url,kantlipsum}
\usepackage{lscape}
\usepackage{array}
\usepackage{booktabs}
\usepackage{txfonts}
\usepackage{colortbl}%
  \newcommand{\myrowcolour}{\rowcolor[gray]{0.925}}
\newenvironment{Figure}
  {\par\medskip\noindent\minipage{\linewidth}}
  {\endminipage\par\medskip}
  
\DeclareCaptionFont{white}{\color{white}}
\DeclareCaptionFormat{listing}{\colorbox[RGB]{60,100,180}{\parbox{0.40\textwidth - 2 \fboxsep}{\hspace{8pt}#1#2#3}}}
\captionsetup[lstlisting]{format=listing,labelfont=white,textfont=white, singlelinecheck=false, margin=0pt, font={bf,footnotesize}}

%\widowpenalties 1 10000
%\raggedbottom

\includeonly{texfiles/planning.tex}

\journal{Systems and Software}

\begin{document}

\begin{frontmatter}

%% Title, authors and addresses

\title{Improving Load, Performance and Stress Search Based Testing using a Hybrid Metaheuristic Approach}
%collaborative approach}

%% use the tnoteref command within \title for footnotes;
%% use the tnotetext command for the associated footnote;
%% use the fnref command within \author or \address for footnotes;
%% use the fntext command for the associated footnote;
%% use the corref command within \author for corresponding author footnotes;
%% use the cortext command for the associated footnote;
%% use the ead command for the email address,
%% and the form \ead[url] for the home page:
%%
%% \title{Title\tnoteref{label1}}
%% \tnotetext[label1]{}
%% \author{Name\corref{cor1}\fnref{label2}}
%% \ead{email address}
%% \ead[url]{home page}
%% \fntext[label2]{}
%% \cortext[cor1]{}
%% \address{Address\fnref{label3}}
%% \fntext[label3]{}


%% use optional labels to link authors explicitly to addresses:
%% \author[label1,label2]{<author name>}
%% \address[label1]{<address>}
%% \address[label2]{<address>}


\author[serpro,unifor,unifor,serpro]{Francisco Nauber Bernardo Gois,Pedro Porf\'irio Muniz de Farias,Andr\'e Lu\'is Vasconcelos Coelho, Thiago Monteiro Barbosa}

\address[serpro]{Serviço Federal de Processamento de Dados,
Avenida Pontes Vieria ,832, Fortaleza, Cear\'a 60130-240}
\address[unifor]{Universidade de Fortaleza, Avenida Pontes Vieria ,832, Fortaleza, Cear\'a 60130-240}

\begin{abstract}
Some software must respond to  thousands or millions of concurrent requests. 
These systems must be properly tested to ensure that they can function correctly under the expected load.  
Load, Performance and Stress Based Testing aims to find test scenarios which produce execution times violating the timing constraints specified. 
The purpose of this paper is the is to identify  test scenarios that exercise a system in a way that tasks are pushed as close as possible to their deadlines. The research proposes  a approach using hybrid metaheuristc in load, performance and stress test models using Genetic Algorithms, Simulated Annealing and Tabu Search Algorithms. A tool named IAdapter, a JMeter Plugin to perform Search Based load, Performance or Stress tests, was developed. Two experiments were conducted to validate the proposed approach. The first experiment has been applied in an emulated component and the second one has been applied in an installed Moodle application. In both experiments, the use of a hybrid metaheuristc has obtained better fitness values.
\end{abstract}


\begin{keyword}
Search Based Testing \sep Tabu Search \sep Hybrid Metaheuristics
%% keywords here, in the form: keyword \sep keyword

%% MSC codes here, in the form: \MSC code \sep code
%% or \MSC[2008] code \sep code (2000 is the default)

\end{keyword}

\end{frontmatter}

%%
%% Start line numbering here if you want
%%
%\linenumbers


\section{Introduction}

Many systems must support concurrent
access by hundreds or thousands of users. The failure to scale users results in catastrophic failures and unfavorable media coverage \cite{Jiang2010}. To assure the quality of these systems, performance, stress and load testing is a required testing procedure\cite{Jiang2009}. 

The explosive growth of the Internet has contributed to  increase the need for applications to perform at an appropriate speed. Performance problems have a bad habit of turning up late in the application life cycle, and the later you discover them,  the greater is the cost to fix them \cite{Molyneaux2009}.
% * <naubergois@gmail.com> 2015-09-16T23:58:29.774Z:
%
%  Alterar frase
%
% ^ <naubergois@gmail.com> 2015-09-16T23:58:46.491Z.
% * <naubergois@gmail.com> 2015-09-16T23:58:52.202Z:
%
% 
%
The use of load testing is an increasingly common practice due to the increasing number of users. In this scenario, the inadequate treatment of a workload generated by concurrent or simultaneous
access, generated by system users, can result in highly critical failures and corrosion of the company's image in their customers' view \cite{Draheim2006b} \cite{Jiang2010}. 
% * <naubergois@gmail.com> 2015-09-16T23:58:52.451Z:
%
%  Frase coloquial
%
% ^ <naubergois@gmail.com> 2015-09-16T23:59:03.275Z.

The Load Testing determines the responsiveness, throughput, reliability or scalability of a system under a given workload. The quality of the results of system's load tests is closely linked to the implementation of the workload strategy. The performance of many applications depends on the load applied under different conditions. In some cases, performance degradation and failures arise only in stress conditions \cite{Garousi2010} \cite{Jiang2010}.

A  Load, Performance or Stress tests uses set of workloads that consists of many types of usage scenarios and different user numbers combinations. A load is typically based on an operational profile. Different parts of an application should be tested on various parameters and stress conditions \cite{Babbar2011}. The correct application of a load test should cover most part of application under ordinary conditions (Load or Performance Test) or above the expected load conditions (Stress Test) \cite{Draheim2006b} \cite{Luiz2011} \cite{Fe2004}.



The Fig. \ref{fig:example} shows an example  of a system under test with three pages (Main Page, Profile Page and Search Page) and six possible users. From the combinations of users and application pages various scenarios can be created as the scenarios 1 and 2 presented in the figure. The first scenario presents a test that has passed and the second scenario a test that had an http error 404.

\begin{figure}[ht]
\centering
\caption{Possible test scenarios for a hypothetical application}
\includegraphics[width=0.4\textwidth]{./images/diagram.png}
\label{fig:example}
\end{figure}

A performance test usually lasts for several hours or even a few days and only tests a limited number of workloads. The major challenge is to find the workloads  that exposes a major number of errors and discover the maximum number of users supported by a application under test \cite{Barna2011}. 

Search Based  Test is seen as a promising approach for verifying timing constraints \cite{Afzal2009a}. The main objective of load, performance and stress Search Based Test is to find test scenarios which produce execution times violating the specified timing constraints \cite{Sullivan}. 


The main objective of the research is to find test scenarios that maximize the likelihood of task deadline misses. The Fig. \ref{fig:solution} presents a illustrative example where the presented research work approach finds test scenarios that expose errors or response times above the maximum response time expected. There are many benefits in the use of solution:

\begin{itemize}
\item Generate stress test cases that maximize the likelihood of task deadline misses;
\item Automate the search for failure points of a application under test;
\end{itemize}



\begin{figure}[ht]
\centering
\caption{Illustrative example of use of the presented research work}
\includegraphics[width=0.25\textwidth]{./images/solution.png}
\label{fig:solution}
\end{figure}



% * <naubergois@gmail.com> 2015-09-17T00:49:26.764Z:
%
%  Rever paragrafo abaixo
%
%
The paper proposes the use of a approach using  hybrid metaheuristc  with  Genetic Algorithms, Simulated Annealing and Tabu Search Algorithms  in load, performance and stress evolutionary tests.

A tool named IAdapter, a JMeter Plugin to perform Search Based load, Performance or Stress tests, was developed. Two experiments were conducted to validate the proposed approach. The first experiment has been applied in an emulated component and the second one has been applied in an installed Moodle application.

The remainder of the paper is organized as follows. Section 2 presents a brief introduction in load, performance and stress tests. Section 3 presents concepts about WorkLoad Model. Section 4 presents concepts about Hybrid Metaheuristcs. Section 5 presents the research proposed approach. Section 6 presents The IAdapter tool. Section 7 discusses the related work. The Section 8 shows the results of two experiment applied with IAdapter. Conclusions and further work are presented in Section 9.



\section{Load, Performance and Stress Test}

Load, performance and stress testing is typically done to locate bottlenecks in a system, to support a performance tuning effort and to collect other performance-related indicators to help stakeholders get informed about the quality of the application being tested \cite{Sandler2004} \cite{Corporation2007}. 

The Performance Test aims at verifying a specified system performance. This kind of test is executed by simulating hundreds or more simultaneous users  over a defined time interval \cite{DiLucca2006}. The purpose of this test is to demonstrate that the system  reaches its performance objectives \cite{Sandler2004}. 


In load tests, the system is evaluated in pre-defined load levels \cite{DiLucca2006}. The aim of this test is to reach the performance targets for availability, concurrency, throughput and response time of the system. Load Test is the closest to real application use \cite{Molyneaux2009}.

Stress test verifies the system behaviour against heavy workloads \cite{Sandler2004}, being executed to evaluate a system beyond its limits, validate system response in activity peaks and verify if the system is able from recover from these conditions. They differ from other kinds of testing  because the system is executed on or beyond its breakpoints, forcing the application or the supporting infrastructure to fail \cite{DiLucca2006} \cite{Molyneaux2009}.

Automated tools are needed to carry out serious load, stress and performance testing. Sometimes , there is simply no practical way to provide reliable, repeatable performance tests without using some form of automation. The aim of any automated test tool is to simplify the testing process \cite{Molyneaux2009}.

In the context of  testing, a scenario is a sequence of steps in your application. It can represent a use case or a business function such as searching a product catalog, adding an item to a shopping cart or placing an order \cite{Corporation2007}. 

Load, Performance and Stress results are measured by indicators. Some researchers advocate the 90-percentile response time is a better measurement than the average/medium response time, as the 90-percentile accounts for most of the peaks, while eliminating the outliers \cite{Jiang2010}.


%% Text of abstract
Some software must respond to  thousands or millions of concurrent requests. 
These systems must be properly tested to ensure that they can function correctly under the expected load.  
Load, Performance and Stress Based Testing aims to find test scenarios which produce execution times violating the timing constraints specified. 
The purpose of this paper is the is to identify  test scenarios that exercise a system in a way that tasks are pushed as close as possible to their deadlines. The research proposes  a approach using hybrid metaheuristc in load, performance and stress test models using Genetic Algorithms, Simulated Annealing and Tabu Search Algorithms. A tool named IAdapter, a JMeter Plugin to perform Search Based load, Performance or Stress tests, was developed. Two experiments were conducted to validate the proposed approach. The first experiment has been applied in an emulated component and the second one has been applied in an installed Moodle application. In both experiments, the use of a hybrid metaheuristc has obtained better fitness values.


\section{WorkLoad Model}

Load, Performance or Stress testing projects should start with the development of a model for user workload that an application receives. This should take into consideration various performance aspects of the application and the infrastructure that a given workload will impact. A workload is a key component of such a model.

The term Workload represents  the size of the demand that will be imposed on the application under test in an execution. The metric unit used for define a Workload is dependent on the application domain, such as the length of the video in a transcoding application of multimedia files or the size of the input files to a file compression application \cite{Feitelson2013} \cite{Molyneaux2009} \cite{Goncalves2014}. 

Workload is also defined by the distribution of load between the identified transactions at a given time. Workload helps us study the system behavior identified in several load model. Workload model can be designed for verify predictability, repeatability and scalability of a system \cite{Feitelson2013} \cite{Molyneaux2009}.


Workload modeling is the try to create a simple and general model, which can
then be used to generate synthetic workloads. The goal is typically to be able to create workloads that can
be used in performance evaluation studies. Sometimes, the synthetic workload is supposed to be
similar to those that occur in practice on real systems \cite{Feitelson2013} \cite{Molyneaux2009}.

There are two kinds of Workload models: descriptive and generative. The difference is that descriptive models just try to mimic the phenomena observed in the workload, whereas generative models try to emulate the process that generated the workload in the first place \cite{DiLucca2006}. 

On descriptive models, one finds different levels of abstraction on one hand, and different levels of faithfulness to the original data on the other hand. The
most strictly faithful models try to mimic the data directly using statistical distribution of data. The most common strategy used in descriptive modeling is to create a statistical
model of an observed workload (Fig. \ref{fig:descriptivemodel}). This model is applied to all the workload
attributes, e.g. computation, memory usage, I/O behavior, communication, etc \cite{DiLucca2006}. The Fig. \ref{fig:descriptivemodel} shows a simplified workflow of a descriptive model. The workflow has six phases. In first phase, the user uses the system in the production environment. In second phase, the tester collects user's data, like logs, clicks and preferences, in the system . The third phase consists in developing a model to emulate the user's behaviour. The fourth phase is made up of the execution of the test, emulation of the user's behaviour and log's gathering.


\begin{figure}[!ht]
\centering
\includegraphics[width=0.4\textwidth]{./images/workloadmodel1.png}
\caption{Workload modeling based on statistical data \cite{DiLucca2006}}
\label{fig:descriptivemodel}
\end{figure}

Generative models are indirect, in the sense that they do not model the statistical distributions. Instead, they describes how users will behave and when they generate the workload. An important benefit of the generative approach is
that it facilitates manipulations of the workload. It is often desirable to be able to change the workload conditions as part of the evaluation. Descriptive models do not offer any option regarding how to do so. But with generative models, we can modify the workload-generation process to fit the desired conditions \cite{DiLucca2006}. The difference between the workflows of descriptive and generative models is that user behavior is not collected from logs, but simulated from a model that can receive feedback from the test execution (Fig. \ref{fig:generativemodel}).

\begin{figure}[!ht]
\centering
\includegraphics[width=0.4\textwidth]{./images/workloadmodel2.png}
\caption{Workload modeling based on Generative Model \cite{DiLucca2006}}
\label{fig:generativemodel}
\end{figure}
% * <naubergois@gmail.com> 2015-09-17T00:36:43.958Z:
%
%  Olhar but no começo da frase
%


\section{Hybrid Metaheuristc}


A large  number of researchers have recognized the advantages and huge potential building
hybrid mathematical programming methods and metaheuristics.
The main motivation to create hybrid Metaheuristics is to exploit the complementary character of different optimization strategies. In fact, choosing an adequate combination of algorithmic can be the key for achieving top performance in solving many hard optimization problems \cite{Puchinger2005} \cite{Blum2012}.


There are two main categories of metaheuristc combinations: Collaborative Combinations and Integrative Combinations. The Fig. \ref{fig:metaheuristc} presents the two main categories of Hybrid MetaHeuristc \cite{Raidl2006}.

\begin{figure}[h]
\includegraphics[width=0.5\textwidth]{./images/metaheuristc2.png}
\caption{Categories of metaheuristc combinations \cite{Puchinger2005} }
\label{fig:metaheuristc}
\end{figure}

Collaborative Combinations uses a approach where the  algorithms exchange information, but are not part of each other. In this approach,  algorithms may be executed sequentially or in parallel. The presented research work uses a type of  Collaborative Combination with Sequential Execution.


\section{Improving Load, Performance and Stress Search Based Testing using a Hybrid Metaheuristic Approach}


The proposed solution makes it possible to create a generative model that evolves during the test. The proposed solution model uses Genetic Algorithm, Tabu Search and Simulated Annealing in two different approaches.  The first approach uses the three algorithms independently and the second approach uses the three algorithms collaboratively (Hybrid Metaheuristic approach).

In the first approach , the algorithms do not share their best individuals among themselves. Each algorithm evolves in a separate way (Fig. \ref{fig:firstaproach}). The second approach use the algorithms in a collaborative mode (Hybrid Metaheuristic). In this approach, the three algorithms share their best individuals found (Fig. \ref{fig:secondapproach}).

The next subsections present details about the used metaheuristcs algorithms (genotype representation and fitnesse function) and the IAdapter components.

\begin{figure}[h]
\includegraphics[width=0.5\textwidth]{./images/independ.png}
\caption{Use of the algorithms independently}
\label{fig:firstaproach}
\end{figure}
\begin{figure}
\includegraphics[width=0.5\textwidth]{./images/collaborative.png}
\caption{Use of the  algorithms collaboratively}
\label{fig:secondapproach}
\end{figure}

\subsection{Genotype representation}

The Genotype representation is composed by a linear vector with 23 genes. The first gene represents the name of individual. The second gene presents the  algorithm (Genetic Algorithm, Simulated Annealing or Tabu Search) used by the individual. The third gene represents the type of test (Load, Stress or Performance). Next genes represent 10 scenarios and their numbers of users. Each scenario is an atomic operation, the scenario must log in the application, run the task goal and undo any changes performed, returning the application to it's original state. 

The Fig. \ref{fig:genomarepresentation} presents the genome representation and  a example using the crossover operation. In the example, the genotype 1 has the Login scenario with 2 users; the Form scenario with 0 users and the Search scenario with 3 users. The genotype 2 has the Delete scenario with 10 users; the Search scenario with 0 users and the Include scenario with 5 users. After the crossover operation, We obtain a genotype with  Login scenario with 2 users; the Search scenario with 0 users and the Include scenario with 5 users.

\begin{figure}[h]
\includegraphics[width=0.5\textwidth]{./images/genomerepresentation.png}
\caption{Genotype representation and crossover example}
\label{fig:genomarepresentation}
\end{figure}

The Fig. \ref{fig:neighbourtaby} shows the strategy used by the IAdapter to obtain the genotype of the neighbours for the Tabu Search and Simulated Annealing algorithms.  The neighbours are obtained by the modification of a single cromossome (scenario or  number of users) in the genotype.

\begin{figure}[h]
\includegraphics[width=0.5\textwidth]{./images/TabuNE.png}
\caption{Tabu Search and Simulated Annealing neighbour strategy}
\label{fig:neighbourtaby}
\end{figure}


\subsection{Initial population}

The strategy use by the plugin to instantiate the initial population is to generate 50\% of the individuals randomically and 50\% of the initial population are distributed in three  ranges of values:

\begin{itemize}
\item 30\% of the maximum allowed users in the test ;
\item 60\% of the maximum allowed users in the test; and
\item 90\% of the maximum allowed users in the test.
\end{itemize}


\subsection{Objective (Fitnesse) Function}

The proposed solution was designed to be used with the independent testing teams in various situations where the team has no direct access to the environment where the application under test was installed. Therefore,  The IAdapter uses a measurement approach to the definition of the fitnesse function. The fitnesse function applied to IAdapter solution is governed by the following equation:

\begin{equation}
\begin{aligned}
fit=90percentileweigth* 90percentiletime\\
+80percentileweigth*80percentiletime\\+
70percentileweigth*70percentiletime+\\
maxResponseWeigth*maxResponseTime+\\
numberOfUsersWeigth*numberOfUsers-penalty
\end{aligned}
\end{equation}

The proposed solution's fitnesse function uses a series of adaptable user-defined weights ( 90percentileweigth, 80percentileweigth,  70percentileweigth, maxResponseWeigth and numberOfUsersWeigth). These weights make it possible to customize the search plugin functionality. The penalty is applied when a application under test responds in a longer time than the level of service.


%\FloatBarrier
\section{IAdapter}

IAdapter is a JMeter Plugin to perform Search Based Load, Performance or Stress Tests. JMeter is a desktop application, designed to test and measure the performance and functional behavior of applications \cite{Nevedrov2007}. The IAdapter plugin implements the solution proposed in the section 6

The JMeter have components organized  in a hierarchical manner. The IAdapter plugin provides three main components: WorkLoadThreadGroup, WorkLoadSaver, and WorkLoadController.
 
The WorkLoadThreadGroup is a component that creates an initial population and configure the algorithms used in IAdapter . The Fig. \ref{fig:tela1iadapter} presents the main screen of the WorkLoadThreadGroup component. The component has a name \ding{202}, a set of configuration tabs \ding{203}, a list of individuals by generation \ding{204}, a button to generate an initial population \ding{205} and a button to export the results \ding{206}.

\begin{figure}[h]
\includegraphics[width=0.5\textwidth]{./images/tela1iadapter.png}
\caption{WorkLoadThreadGroup component}
\label{fig:tela1iadapter}
\end{figure}

The WorkLoadSaver component is responsible for saving all data in the database. The operation of the component only requires its inclusion in the test script.

The WorkLoadController represents a scenario of test. The WorkLoadController represents a scenario of test. All actions necessary to test a application should be included in this component. All instance of the component need to login in the application under test and return the application to it's original state.

\section{Related Work}

The search for the longest execution time is regarded as a discontinuous, nonlinear, optimization problem, with the input domain of the system under test as search space \cite{Sullivan}. The main objective of search based testing in performance,stress and load tests is to find test scenarios which produce execution times violating the timing constraints specified. If a temporal error is found, the test was successful \cite{Sullivan}. The application of evolutionary algorithms to load, performance and stress tests involves finding the best and worst case execution times (BCET, WCET) to determine if timing constraints are fulfilled \cite{Afzal2009a}. 

Some Search Based Tests uses a cost (fitnesse) function to select the best individuals. There has two measurement units normally associated with the fitnesse function in load, performance or stress test: Processor Cycles and Execution Time. The Processor Cycles approach describes a fitness function in terms of processor cycles. The Execution Time approach involves executing the application under test measuring the execution time \cite{Afzal2009} \cite{tracey2000search}.
% * <naubergois@gmail.com> 2015-09-17T01:17:52.488Z:
%
%  Rever esse paragrafo
%
The Table \ref{tab:comparison}  shows a comparison between the presented research work and the load, performance and stress test researches presented by Afzal et. al. \cite{Afzal2009}. Afzal's work was added with some of the latest research in the area (\cite{Garousi2006} \cite{Garousi2010} \cite{DiAlesio2013} \cite{DiAlesio2014} \cite{Alesio2015}). 


The columns represents the type of tool used ( Prototype or Functional Tool )  and the rows presents the metaheuristic used by each research (Genetic Algorithm, Tabu Search, Simulated Annnealing or a Customized Algorithm). The Table also divides the researches by the type of function fitnesse (Execution Time or Processor Cycles). Most research is limited to making prototypes on genetic algorithms. The presented research work is distinguished from others by having a functional tool using a hybrid approach. 



% * <naubergois@gmail.com> 2015-09-17T01:22:09.554Z:
%
%  Customized Algorithm
%

%\begin{figure}[h]
%\centering
%\includegraphics[width=0.5\textwidth]{./images/comparativo1.png}
%\caption{
%Distribution of the researches over range of applied metaheuristics}
%\label{fig:comparison}
%\end{figure}

% Please add the following required packages to your document preamble:
% \usepackage[table,xcdraw]{xcolor}
% If you use beamer only pass "xcolor=table" option, i.e. \documentclass[xcolor=table]{beamer}
\begin{table}[h]
\centering
\caption{Distribution of the researches over range of applied metaheuristics}
\label{tab:comparison}
\begin{tabular}{p{1.2cm}|p{1.8cm}|p{1.8cm}|p{1.8cm}|}
\cline{2-4}
                                                                & \multicolumn{2}{c|}{\textbf{Prototypes}}            & \textbf{Functional Tool} \\ \cline{2-4} 
                                                                & \begin{minipage}{0.2\textwidth}\footnotesize Execution Time  \end{minipage}          & \begin{minipage}{0.2\textwidth}\footnotesize Processor Cycles \end{minipage}        & \begin{minipage}{0.2\textwidth}\footnotesize Execution Time \end{minipage}           \\ \cline{2-4} 
%\setlength{\extrarowheight}{20pt}
\begin{tabular}[c]{@{}l@{}}\begin{minipage}{0.1\textwidth}\scriptsize GA + SA  \\ + Tabu \\ Search \end{minipage}\end{tabular}  & \cellcolor[HTML]{FFCCC9} & \cellcolor[HTML]{FFCCC9} & \cellcolor[HTML]{F8FF00} \begin{minipage}{0.2\textwidth} \scriptsize \textbf{  \\ IADAPTER \\ Gois, 2015 \\} \end{minipage}  \\[2ex] \cline{2-4} 
\begin{minipage}{0.1\textwidth}\scriptsize GA \end{minipage}                                                              & \cellcolor[HTML]{CD9934} \begin{minipage}{0.12\textwidth}   \tiny \textnormal{ \\  Alander et al.,1998 \cite{Alander} \\ Wegener et al., 1996 and 1997 \cite{Wegener1997}\cite{J.WegenerK.GrimmM.GrochtmannH.Sthamer1996} \\  Sullivan et al., 1998 \cite{Sullivan} \\ Briand et al., 2005 \cite{Briand2005} \\ Canfora et al., 2005 \cite{Canfora}  \\ }\end{minipage} & \cellcolor[HTML]{CD9934} \begin{minipage}{0.12\textwidth} \tiny \textrm{  \\ Wegener and Grochtmann, 1998 \cite{Wegener1998} \\  Mueller et al., 1998 \cite{Mueller1998} \\ Puschner et al. \cite{Puschner1998} \\ Wegener et al., 2000 \cite{Stations} \\ Gro et al., 2000 \cite{Gross2000}  \\ }\end{minipage}& \cellcolor[HTML]{CD9934} \begin{minipage}{0.12\textwidth}   \tiny \textnormal{ \\  Di Penta, 2007 \cite{Penta2007} \\ Garoussi, 2006 \cite{Garousi2006} \\ Garousi, 2008 \cite{Garousi2008} \\ Garousi, 2010 \cite{Garousi2010} \\ } \end{minipage} \\[2ex] \cline{2-4} 
\begin{minipage}{0.1\textwidth}\scriptsize Simulated \\ Annealing \\ (SA) \end{minipage}                                                             & \cellcolor[HTML]{FFCCC9} & \cellcolor[HTML]{FFCCC9} & \cellcolor[HTML]{CD9934} \begin{minipage}{0.12\textwidth}   \tiny  Tracey, 1998 \cite{Tracey1998} \end{minipage} \\[2ex] \cline{2-4}
\begin{minipage}{0.1\textwidth}\scriptsize  Constraint \\ Programming \end{minipage}                                                             & \cellcolor[HTML]{FFCCC9} & \cellcolor[HTML]{FFCCC9} & \cellcolor[HTML]{CD9934} \begin{minipage}{0.12\textwidth}   \tiny  Alesio, 2014 \cite{DiAlesio2014} \\ Alesio, 2013 \cite{DiAlesio2013}  \end{minipage} \\[2ex] \cline{2-4} 
\begin{minipage}{0.1\textwidth}\scriptsize  GA +\\ Constraint \\ Programming \end{minipage}                                                             & \cellcolor[HTML]{FFCCC9} & \cellcolor[HTML]{FFCCC9} & \cellcolor[HTML]{CD9934} \begin{minipage}{0.12\textwidth}   \tiny  Alesio, 2015 \cite{Alesio2015} \end{minipage} \\[2ex] \cline{2-4} 
\setlength{\extrarowheight}{20pt}
\begin{tabular}[c]{@{}l@{}}
\begin{minipage}{0.1\textwidth}\scriptsize Customized \\ Algorithm \end{minipage}\end{tabular} & \cellcolor[HTML]{FFCCC9} & \cellcolor[HTML]{CD9934}  \begin{minipage}{0.12\textwidth}   \tiny  \textnormal{   \raggedleft Pohlheim, 1999 \cite{Pohlheim2005}  } \end{minipage} & \cellcolor[HTML]{FFCCC9} \\[4ex] \cline{2-4}
\end{tabular}
%\begin{tabular}[c]{@{}l@{}}
%\begin{minipage}{0.1\textwidth}\scriptsize GA  Constraint Programming \end{minipage}\end{tabular} & \cellcolor[HTML]{FFCCC9} & \cellcolor[HTML]{CD9934}  \begin{minipage}{0.12\textwidth}   \tiny  \textnormal{   \raggedleft Alesio, 2015 \cite{Alesio2015}   } \end{minipage} & \cellcolor[HTML]{FFCCC9} \\[4ex] \cline{2-4}
%\end{tabular}
\end{table}


The presented research work and Alesio's approach \cite{Alesio2015} uses a hybrid approach with a functional tool. The table \ref{tab:alesiogois} presents the main differences between Alesio's and Gois's approaches. While the present research uses an approach based on usage scenarios performing tests on an application installed in an available environment, Alesio use sequence diagrams  to select for arrival time of tasks in Systems from  safety-critical domains. 

\begin{table}[h]
\centering
\caption{Main differences between Alesio's \cite{Alesio2015} and Gois's approaches}
\label{tab:alesiogois}
\begin{tabular}{l|l|l|}
\cline{2-3}
                                                                                  & Alesio et al. \cite{Alesio2015}                                                                                                             & Gois et al.                                                                                                                            \\ \hline
\multicolumn{1}{|l|}{Metaheuristcs}                                               & \begin{tabular}[c]{@{}l@{}}GA+ \\ Constraint Programming\end{tabular}                                                      & \begin{tabular}[c]{@{}l@{}}GA+SA+\\ Tabu Search\end{tabular}                                                                           \\ \hline
\multicolumn{1}{|l|}{Inputs}                                                      & \begin{tabular}[c]{@{}l@{}}Design Model (Time and \\ Concurrency\\  Information)\end{tabular}                              & \begin{tabular}[c]{@{}l@{}}Number of Users\\ Ramp-up\\ Test scenarios\end{tabular}                                                     \\ \hline
\multicolumn{1}{|l|}{\begin{tabular}[c]{@{}l@{}}Main\\ Objective\end{tabular}}    & \begin{tabular}[c]{@{}l@{}}Find task arrival times\\ of aperiodic tasks that\\  maximizing\\  deadline misses\end{tabular} & \begin{tabular}[c]{@{}l@{}}Find the number of\\  users, ramp-up and \\ test scenarios that\\ maximizing\\ deadline misses\end{tabular} \\ \hline
\multicolumn{1}{|l|}{\begin{tabular}[c]{@{}l@{}}Main \\ Application\end{tabular}} & \begin{tabular}[c]{@{}l@{}}Systems from \\ safety-critical \\ domains\end{tabular}                                         & \begin{tabular}[c]{@{}l@{}}Web and Mobile \\ applications\end{tabular}                                                                 \\ \hline
\end{tabular}
\end{table}



%\nocite{Alander,Sullivan,Wegener1997,Briand2005,Canfora,Wegener1998,Mueller1998,Puschner1998,Wegener1999,Gro,Gross2003,Tlili1917}


\section{Experiments}

This section presents two experiments. The first one has been applied in an emulated component and The second experiment has been applied in an installed Moodle application. The experiments used this fitnesse function:

\begin{equation}
\begin{aligned}
fit=0.9* 90percentiletime\\
+0.1*80percentiletime\\+
0.1*70percentiletime+\\
0.1*maxResponseTime+\\
0.2*numberOfUsers-penalty
\end{aligned}
\end{equation}

The fitnesse function used in the experiments intended to find individuals with the highest percentile of 90\%, followed by individuals with higher percentile time of 80\% and 70\%, maximum response time and number of users.

The first experiment has implemented 27 generations and the second experiment has performed 6 generations, with 300 executions by generation (100 times for each algorithm),  generating 300 new individuals. The experiments had used a initial population of 100 individuals. The Genetic Algorithm used the top 10 individuals from each generation to the crossover operation. The Tabu List has been configured with the size of 10 individuals and expire every 2 generations.  The mutation operation was applied to 10\% of the population on each generation. 

\subsection{First Experiment- Emulated Class Test}

The first experiment aimed to apply performance, load and stress testing in a simulated component. The purpose of using a simulated component is able to perform a greater number of generations in a shorter time available and eliminate variables such as the use of databases and application servers. The first experiment used a test class  named SimulateConcurrentAccess. These class have a static variable named \textit{x} and a set of methods that uses the variable in a synchronized context ( Listing \ref{classsimulated}).

\lstdefinestyle{outline}{
		language=Java,
         basicstyle=\scriptsize\ttfamily,
         numberstyle=\tiny,
         numbersep=5pt,
         tabsize=2,
         extendedchars=true,
         breaklines=true,
         keywordstyle=\color{black}\bf,
         frame=b,  % <<<<<<<<<<<<<<<<<<<<<<<<<<
         stringstyle=\color{green!40!black}\ttfamily,
         showspaces=false,
         showtabs=false,
         numbers=left,
         xleftmargin=17pt,
         framexleftmargin=17pt,
         framextopmargin=1pt, % <<<<<<<<<<<<<<<<<<<<<<
         showstringspaces=false,
         %backgroundcolor=\color[RGB]{200,200,200},
         belowcaptionskip=0pt
}

\begin{lstlisting}[style=outline,caption={SimulateConcurrentAcess class},float,label=classsimulated]
public class SimulateConcurrentAccess {
  @Test
  public void firstScenario() {		
    synchronized (StaticClass.class) {
			for (int i = 0; i <= 1000; i++) {
				StaticClass.x += i;
			}
			StaticClass.x = 0;
		}
	}
	
	  @Test
  public void secondScenario() {		
    synchronized (StaticClass.class) {
			for (int i = 0; i <= 2000; i++) {
				StaticClass.x += i;
			}
			StaticClass.x = 0;
		}
	}
\end{lstlisting}


Fig.\ref{fig:exp1bestresults} presents the best results in 27 generations applied in the first experiment . The Figure shows the results obtained with the algorithms with and without collaboration. The $x$ axis  represents the generation number and the $y$ axis represents the best fitnesse value obtained until the current generation. The results of the experiment showed that the use of cooperation between the three algorithms resulted in find individuals with better fitnesse values.

\begin{figure}[h]
\centering
\caption{Best results obtained in 27 generations}
\includegraphics[width=0.5\textwidth]{./images/generationcomparative.png}
\label{fig:exp1bestresults}
\end{figure}
The table \ref{tab:averagefirst} presents the results obtained by the Hybrid Metaheuristc (HM), Genetic Algorithm (GA), Simulated Annealing (SA) and TABU Search (TS) from 27 generations in the first experiment. The values are the maximum value of the fitnesse obtained in each algorithm. 

\begin{table}[h]
\centering
\caption{Fitnesse function maximum value by algorithm}
\label{tab:averagefirst}
\begin{tabular}{|l|l|l|l|l|}
\hline
GEN & HM & TS  & GA    & SA    \\ \hline
1          & 11238 & 11238         & 11238 & 11238 \\ \hline
2          & 11804 & 11596         & 11801 & 10677 \\ \hline
3          & 11787 & 8932          & 8411  & 10869 \\ \hline
4          & 11723 & 9753          & 9611  & 10760 \\ \hline
5          & 8164  & 9780          & 10738 & 4794  \\ \hline
6          & 11802 & 9781          & 11086 & 6120  \\ \hline
7          & 9985  & 5782          & 11272 & 11798 \\ \hline
8          & 11803 & 11749         & 10084 & 11309 \\ \hline
9          & 11806 & 7284          & 11633 & 10766 \\ \hline
10         & 11807 & 9386          & 11717 & 4557  \\ \hline
11         & 11802 & 9653          & 11802 & 11151 \\ \hline
12         & 11807 & 10594         & 11793 & 9434  \\ \hline
13         & 11802 & 10848         & 10382 & 11805 \\ \hline
14         & 11801 & 11551         & 7219  & 10237 \\ \hline
15         & 11807 & 1701          & 7189  & 9338  \\ \hline
16         & 11813 & 6203          & 11758 & 5321  \\ \hline
17         & 11805 & 10720         & 10805 & 11748 \\ \hline
18         & 9600  & 6371          & 11698 & 7818  \\ \hline
19         & 11733 & 8160          & 11648 & 11509 \\ \hline
20         & 9589  & 9428          & 11805 & 4813  \\ \hline
21         & 11800 & 9463          & 11798 & 10801 \\ \hline
22         & 11805 & 11799         & 11804 & 6029  \\ \hline
23         & 11836 & 11655         & 11800 & 3579  \\ \hline
24         & 11805 & 11512         & 11803 & 5761  \\ \hline
25         & 11804 & 11573         & 11802 & 9680  \\ \hline
26         & 11800 & 11575         & 11403 & 9388  \\ \hline
27         & 11805 & 10691         & 11745 & 9465  \\ \hline
\end{tabular}
\end{table}

The signed-rank Wilcoxon non-parametrical procedure was used for comparing the results. The procedure showed that there was a significant improvement in the results with the collaborative approach.

\subsection{Second Experiment- Moodle Application Test}

The second experiment uses a Moodle application installed in a machine with 500 Gb of hard disk and 8 Gb of memory.The study used six application scenarios:

\begin{itemize}
\item PostDeleteMessage- This scenario post and delete messages in the moodle application.
\item MyHome- This scenario access the user's homepage of the application.
\item Login- This scenario are responsible by the user authentication of the application.
\item Notifications- This scenario enter in the notification page of each user.
\item Start Page- Initial start page of the application.
\item Badge- This scenario enter in the Badge page.
\end{itemize}

The maximum tolerated response time in test was 30 seconds.  Any  individuals that obtained a time longer than the stipulated maximum time suffered penalties.  The whole process of stress and performance tests, which took three days and about 1800 executions, was carried out without the need for monitoring of a test designer. The tool have  selected automatically the next scenarios to be run up to the limit of six generations previously established. 

The Table \ref{tab:secondexperiment} presents the maximum fitnesse value obtained by the Hybrid Metaheuristc (HM), Genetic Algorithm (GA), Simulated Annealing (SA) and TABU Search (TS) in each generation. 

\begin{table}[h]
\centering
\caption{Results obtained from the second experiment}
\label{tab:secondexperiment}
\begin{tabular}{|l|l|l|l|l|}
\hline
GEN & HM    & TS    & GA    & SA    \\
\hline
1          & 32242 & 32242 & 32242 & 32242 \\
\hline
2          & 34599 & 32443 & 26290 & 35635 \\
\hline
3          & 35800 & 34896 & 34584 & 34248 \\
\hline
4          & 35782 & 34912 & 32689 & 25753 \\
\hline
5          & 35611 & 31833 & 34631 & 8366  \\
\hline
6          & 35362 & 35041 & 33397 & 9706 \\
\hline
\end{tabular}
\end{table}


The small number of samples of the experiment is insufficient to give a statistical significance with Wilcoxon procedure. However, it is noted that in 4 of 6 generations, the collaborative approach presented the best values. The experiment succeeded in finding 29 individuals where the maximum time expected by the application was obtained.  The Table. \ref{tab:secondexperiment1} has a example of the six individuals with the highest fit values in the second experiment. The Table shows the fitnesse value (Fit),  the name of scenario (Scenario), the number of users (Users), the percentiles of 90\%,80\% and 70\% (90per, 80per and 70per) in seconds.  

% Please add the following required packages to your document preamble:
% \usepackage{multirow}
\begin{table}[h]
\centering
\caption{Example of individuals obtained in the second experiment}
\label{tab:secondexperiment1}
\begin{tabular}{|p{0.2cm}|l|l|l|p{0.60cm}|p{0.60cm}|p{0.60cm}|}
\hline
Id&Fit&Scenario&Users&90per&80per&70per\\ \hline
\multirow{2}{*}{1} & \multirow{2}{*}{35800} & MyHome        & 31              & \multirow{2}{*}{30} & \multirow{2}{*}{29} & \multirow{2}{*}{10} \\ \cline{3-4}
                   &                        & Badges        & 4               &                     &                     &                     \\ \hline
\multirow{3}{*}{2} & \multirow{3}{*}{35795} & MyHome        & 30              & \multirow{3}{*}{30} & \multirow{3}{*}{29} & \multirow{3}{*}{10} \\ \cline{3-4}
                   &                        & Notifications & 2               &                     &                     &                     \\ \cline{3-4}
                   &                        & Badges        & 2               &                     &                     &                     \\ \hline
\multirow{2}{*}{3} & \multirow{2}{*}{35782} & MyHome        & 32              & \multirow{2}{*}{30} & \multirow{2}{*}{29} & \multirow{2}{*}{10} \\ \cline{3-4}
                   &                        & Badges        & 3               &                     &                     &                     \\ \hline
\multirow{3}{*}{4} & \multirow{3}{*}{35773} & MyHome        & 22              & \multirow{3}{*}{30} & \multirow{3}{*}{29} & \multirow{3}{*}{10} \\ \cline{3-4}
                   &                        & Notifications & 6               &                     &                     &                     \\ \cline{3-4}
                   &                        & Badges        & 9               &                     &                     &                     \\ \hline
\multirow{2}{*}{5} & \multirow{2}{*}{35771} & MyHome        & 28              & \multirow{2}{*}{30} & \multirow{2}{*}{29} & \multirow{2}{*}{9}  \\ \cline{3-4}
                   &                        & Badges        & 6               &                     &                     &                     \\ \hline
\multirow{2}{*}{6} & \multirow{2}{*}{35683} & MyHome        & 27              & \multirow{2}{*}{30} & \multirow{2}{*}{29} & \multirow{2}{*}{8}  \\ \cline{3-4}
                   &                        & Badges        & 10              &                     &                     &                     \\ \hline
\end{tabular}
\end{table}


The Table. \ref{fig:gened} presents the percentage of genes in all test scenarios by generation with and without colaboration. Most of genes converges to the MyHome feature which has the highest application response time.

% Please add the following required packages to your document preamble:
% \usepackage[table,xcdraw]{xcolor}
% If you use beamer only pass "xcolor=table" option, i.e. \documentclass[xcolor=table]{beamer}
\begin{table}[h]
\centering
\caption{Percentage of genes in each scenario by generation }
\label{fig:gened}
\begin{tabular}{c|c|c|c|c|c|c|c|}
\hline
\rowcolor[HTML]{009901} 
\multicolumn{1}{|c|}{\cellcolor[HTML]{FFFFFF}\textbf{Gen/}}   & \multicolumn{7}{c|}{\cellcolor[HTML]{009901}\textbf{Non collaboration approach}}                                                                                                                                                       \\ \cline{2-8} 
\multicolumn{1}{|c|}{\textbf{Scenarios}}                      & \cellcolor[HTML]{F8FF00}Initial & \cellcolor[HTML]{F8FF00}1 & \cellcolor[HTML]{F8FF00}2 & \cellcolor[HTML]{F8FF00}3 & \cellcolor[HTML]{F8FF00}4 & \cellcolor[HTML]{F8FF00}5 & \cellcolor[HTML]{F8FF00}6                                \\ \hline
\multicolumn{1}{|c|}{\cellcolor[HTML]{009901}Badges}          & 20                              & 18                        & 16                        & 24                        & 15                        & 16                        & 17                                                       \\ \hline
\rowcolor[HTML]{F8FF00} 
\multicolumn{1}{|c|}{\cellcolor[HTML]{F8FF00}\textbf{MyHome}} & \textbf{15}                     & \textbf{59}               & \textbf{55}               & \textbf{48}               & \textbf{53}               & \textbf{50}               & \multicolumn{1}{l|}{\cellcolor[HTML]{F8FF00}\textbf{52}} \\ \hline
\multicolumn{1}{|c|}{\cellcolor[HTML]{009901}StartPage}       & 15                              & 10                        & 12                        & 11                        & 20                        & 18                        & \multicolumn{1}{l|}{19}                                  \\ \hline
\multicolumn{1}{|c|}{\cellcolor[HTML]{009901}Notifications}   & 25                              & 5                         & 11                        & 10                        & 9                         & 10                        & \multicolumn{1}{l|}{9}                                   \\ \hline
\multicolumn{1}{|c|}{\cellcolor[HTML]{009901}Post}            & 8                               & 3                         & 1                         & 3                         & 1                         & 2                         & \multicolumn{1}{l|}{1}                                   \\ \hline
\multicolumn{1}{|c|}{\cellcolor[HTML]{009901}Login}           & 17                              & 5                         & 5                         & 4                         & 2                         & 4                         & \multicolumn{1}{l|}{2}                                   \\ \hline
\multicolumn{1}{l|}{}                                         & \multicolumn{7}{c|}{\cellcolor[HTML]{009901}\textbf{Collaboration approach}}                                                                                                                                                           \\ \hline
\multicolumn{1}{|c|}{\cellcolor[HTML]{009901}Badges}          & 20                              & 29                        & 16                        & 25                        & 9                         & 16                        & 9                                                        \\ \hline
\rowcolor[HTML]{F8FF00} 
\multicolumn{1}{|c|}{\cellcolor[HTML]{F8FF00}\textbf{MyHome}} & \textbf{15}                     & \textbf{29}               & \textbf{69}               & \textbf{49}               & \textbf{74}               & \textbf{66}               & \textbf{76}                                              \\ \hline
\multicolumn{1}{|c|}{\cellcolor[HTML]{009901}StartPage}       & 15                              & 22                        & 10                        & 21                        & 10                        & 10                        & 8                                                        \\ \hline
\multicolumn{1}{|c|}{\cellcolor[HTML]{009901}Nofications}     & 25                              & 10                        & 1                         & 1                         & 2                         & 1                         & 3                                                        \\ \hline
\multicolumn{1}{|c|}{\cellcolor[HTML]{009901}Post}            & 8                               & 2                         & 1                         & 1                         & 1                         & 2                         & 1                                                        \\ \hline
\multicolumn{1}{|c|}{\cellcolor[HTML]{009901}Login}           & 17                              & 8                         & 3                         & 3                         & 4                         & 5                         & 3                                                        \\ \hline
\end{tabular}
\end{table}


%\begin{figure}[h]
%\centering
%\caption{Percentage of genes in all test scenarios by generation }
%\includegraphics[width=0.5\textwidth]{./images/gened.png}
%\label{fig:gened}
%\end{figure}

\section{Conclusion}

This paper presented a approach of use Hybrid Metaheuristc in load, performance and stress testing. Two experiments were performed to validate the solution. The first experiment has been applied in an emulated component and the second experiment has been applied in an installed Moodle application.  The collaborative approach has obtained better fit values in both experiments. 

The main contributions of the research are: The presentation of an approach that uses a hybrid metauristc to perform load, performance and stress tests;  The development of a JMeter plugin  to Search Based Tests and  the automation of the  load, performance or stress test execution process. Among the future work of the research, we can highlight the use of new combinatorial optimization algorithms such as Very large-scale neighborhood search. 





%\section*{Reference}

%% The Appendices part is started with the command \appendix;
%% appendix sections are then done as normal sections
%% \appendix

%% \section{}
%% \label{}

%% References
%%
%% Following citation commands can be used in the body text:
%% Usage of \cite is as follows:
%%   \cite{key}          ==>>  [#]
%%   \cite[chap. 2]{key} ==>>  [#, chap. 2]
%%   \citet{key}         ==>>  Author [#]

%% References with bibTeX database:

\section*{Reference}

\bibliographystyle{model1-num-names}
%\bibliographystyle{plain}
\bibliography{sample}

%% Authors are advised to submit their bibtex database files. They are
%% requested to list a bibtex style file in the manuscript if they do
%% not want to use model1-num-names.bst.

%% References without bibTeX database:

% \begin{thebibliography}{00}

%% \bibitem must have the following form:
%%   \bibitem{key}...
%%

% \bibitem{}

% \end{thebibliography}


\end{document}

%%
%% End of file `elsarticle-template-1-num.tex'.

